\documentclass[12pt, a4paper]{article}
\usepackage[utf8]{inputenc}
\usepackage{graphicx} % Images
\usepackage{float} % Exact image positioning
\usepackage[bottom]{footmisc} % Footnotes will stick to the bottom of the page, AS THEY SHOULD
\usepackage{amsmath}
\usepackage{amssymb}
\usepackage{empheq}
\usepackage{xcolor} % Colored text
\usepackage{minted} % Simple syntax highlighting
%* * * * * * * * * * * * * * * * * *%

% Document margins
\usepackage{geometry}
\geometry
{
	a4paper     ,
	left=3.0cm  ,
 	top=3.5cm   , 
 	right=2.0cm ,
 	bottom=2.0cm
}

% Indents the first paragraph
\usepackage{indentfirst}

% Indents all paragraphs
\setlength{\parindent}{1.25cm}

% Whitespace before each paragraph
\setlength{\parskip}{1em}

% Space between paragraphs
\renewcommand{\baselinestretch}{1.2}

% Vertical space between lines inside \align
\setlength{\jot}{10pt}

\title{Computação Gráfica\\ Sítense}
\author{Lucas Moura de Carvalho\\9862905}
\date{}

% Document start
\begin{document}

% Changes the title of the content list
\renewcommand*\contentsname{Sumário}

% Removes page counting
\pagenumbering{gobble}
% Inserts the tittle
\maketitle
\newpage
\tableofcontents
\newpage

% Restores page conting
\pagenumbering{arabic}

\section{Tela}

Consideraremos $\Omega = (\Omega_{x},\;\Omega_{y},\;\Omega_{z})$ como a posição do observador, olhando na direção de um ponto $D = (D_{x},\;D_{y},\;D_{z})$. O vetor principal que determina a direção do centro da tela é vetor $P = \overrightarrow{OD} = D-O$, normalizado para $\hat{P}=\frac{P}{||P||}$.

O vetor orthogonal na direção horizontal é $\hat{H} = \hat{P} \times (0,0,1)$ e o vetorl orthognal na direção vertical é $\hat{V} = -(\hat{P} \times \hat{H})$. Com esses três vetores é possível gerar pontos nas direções orthogonais do plano projetivo a partir de movimentos ao longo de ambos os vetores. Considere o centro do plano a uma distância d, então um ponto gerado por movimentos orthogonais a partir do centro em disâncias $\delta_{h}$ na horizontal e $\delta_{v}$ na vertical é dado por:
\begin{align*}
	& p(x,\;y,\;d,\;\delta_{h}, \delta_{v}) = O + d \cdot T + \delta_{h}\hat{H} + \delta{v}\hat{V}
\end{align*}


\end{document}